\documentclass[journal]{IEEEtran}
% *** CITATION PACKAGES ***
\usepackage{cite}

% *** GRAPHICS RELATED PACKAGES ***
%
\usepackage[pdftex]{graphicx}

% *** MATH PACKAGES ***
\usepackage{amsmath}
\interdisplaylinepenalty=2500

% *** SPECIALIZED LIST PACKAGES ***
% to write algorithms in the corpus of the report
\usepackage{algorithmic}

% *** PDF, URL AND HYPERLINK PACKAGES ***
\usepackage{url}

% correct bad hyndenation here
\hyphenation{op-tical net-works semi-conduc-tor}

% *** TABLES ***
\usepackage[para]{threeparttable}
\usepackage{lipsum} % filler text
\usepackage{float}
\usepackage{tabularx}
\usepackage{makecell}

% to use \degree
\usepackage{gensymb}

% to use \nth 
\usepackage[super]{nth}



\begin{document}
\lipsum[1] % filler text

\title{Dual-Objective Scheduling of Rescue Vehicles to\\
Distinguish Forest Fires via Differential Evolution\\
and Particle Swarm Optimization\\
Combined Algorithm}

\author{Guidolin~Davide,
        Guglielmi~Matteo}% 

% The paper headers
\markboth{University of Trento, Bio-Inspired AI, September~6}%
{Shell \MakeLowercase{\textit{et al.}}: Implementation of the paper "Dual-Objective Scheduling of Rescue Vehicles to 
Distinguish Forest Fires via Differential Evolution
and Particle Swarm Optimization
Combined Algorithm"}
% make the title area
\maketitle

% As a general rule, do not put math, special symbols or citations
% in the abstract or keywords.
\begin{abstract}
With the increasing issue of global warming, the problem of forest fires during summer seasons is becoming more severe every year.
For this reason we decided to focus our attention on a project that could possibly deal with this problem. Our attention landed on the paper 
\textit{"Dual-Objective Scheduling of Rescue Vehicles to Distinguish Forest Fires via Differential Evolution and Particle Swarm Optimization Combined Algorithm"}
written by \textit{Guangdong Tian, Yaping Ren, and MengChu Zhou, Fellow, IEEE}. 
In this paper the authors present a method to optimize the fire distinguish time and the number of vehicles used to distinguish a set of fires and their approach is applied to a real-world scenario in Mt. Daxing’anling, China.
The focus of our project is the implementation and testing of their approach.
\end{abstract}

% Note that keywords are not normally used for peerreview papers.
\begin{IEEEkeywords}
    PSO, DE, NSGA-II, Pareto Solutions, Genetic Operators, MHDP
\end{IEEEkeywords}

\IEEEpeerreviewmaketitle

\section{Introduction}
The problem of forest fires is becoming a big issue all around the world. With the continuous rise in temperature and with the less frequent rains in summer, the number of forest fires is increasing every year. However, the number of rescue vehicles is limited and, in case of multiple fire points, deciding how many vehicles to use for each fire point is a difficult task that has to be solved very quickly. In particular, different fire points may have different weather characteristics, like the temperature and the wind speed, and different terrain characteristics, like the slope and the type of terrain, and these parameters have to be taken into account during the decision of the number of vehicles for each fire point. Finally, the distance of the fire point to the fire department and the time that each vehicle takes to extinguish a fire are very important parameters.\\
In the paper \textit{Dual-Objective Scheduling of Rescue Vehicles to Distinguish Forest Fires via Differential Evolution and Particle Swarm Optimization Combined Algorithm}\cite{fire_distinguish}, by Tian, G., Ren, Y. \& Zhou, M., the authors present a  Multi-objective Hybrid Differential-evolution and Particle-swarm-optimization (MHDP) algorithm to minimize the time spent to extinguish a fire while minimizing the total number of vehicles used.

\section{Problem Statement}
Spiegazione fitness functions
\section{MHDP Algorithm}
The Multi-objective Hybrid Differential-evolution and Particle-swarm-optimization algorithm combines DE and PSO into a multi-objective optimization algorithm. 
This algorithm is composed of 5 tasks : Solution Coding, Population Initialization, Evaluation of individuals, Mutation and Crossover.
\subsection{Solution Coding}
Each solution is encoded into a vector of $N$ elements whose $i-$th element indicates the number of vehicles sent to the fire point at location $i$.
\subsection{Generating the initial population}
Each individual is created by generating $N$ random integer values, where the random values are chosen in the range \([L_i, U_i]\) to comply with constraint (\ref{eqn:constraints_a}). 
This procedure is repeated until the individual comply also with constraint (\ref{eqn:constraints_b}).
$PopSize$ individuals are generated.
\subsection{Calculating Fitness Values and Screening Pareto Solutions}
After having generated the initial population we compute the fitness value of each solution and we update the personal best \((P_{best})\) value of each individual and the global best value \((G_{best})\).
Subsequently, we seek for Pareto non-dominated solutions (based on fitness value) and we insert the first front in an archive \(A(g)\) responsible for containing the best solutions found until generation \(g\).
Finally, we screen again the Pareto front in the archive because after the insertion of the new individuals there may be new domination relations between the fresh Pareto Solutions and $A(g-1)$.
\subsection{Mutation and Crossover}
A 3-parent mutation strategy is used:
\begin{equation}
    \label{eqn:differential_operator}
    X_i(g+1) = X_i(g) + F \cdot (X_j(g) - X_k(g))
\end{equation}
where \(F\) is the scaling factor (\(F \in (0, 2)\)), \(X_i(g)\) is the $i-th$ individual in the current population, $i, j, k = 1, 2, ..., PopSize$ and $i \neq j \neq k$.\\
MHDP integrates (\ref{eqn:differential_operator}) into PSO:
\begin{equation}
    \begin{aligned}
    \label{eqn:pso}
    X_i(g+1) = X_i(g) + \Phi[r_1(G_{best} - X_i(g)) \\
    + r_2(P_{best} - X_i(g)) + F(X_j(g) - X_k(g))]
    \end{aligned}
\end{equation}
where $\Phi$ is the $round$ function and $r_1, r_2$ are two random numbers
from a uniform distribution over $[0, 1]$. Here each individual takes a small
step towards $G_{best}$ and a small step towards $P_{best}$. The mutation contribution
is given by  $F(X_j(g) - X_k(g))$ where $X_j(g)$ and $X_k(g)$ are randomly selected.\\
The crossover operation is applied to each individual $X_i$ by selecting another random 
individual $X_k$ in the population and performing the operation $X_{ij} = X_{kj}$ with probability $P_c$ for each $j = 1, 2, ..., N$.\\
After mutation and crossover we adjust the new individuals to meet constraints (\ref{eqn:constraints_a}) and (\ref{eqn:constraints_b}).
\subsection{Updating the archive set}
After mutation and crossover we evaluate again the solutions, we update $P_{best}$ for each individual, $G_{best}$ and the archive $A(g)$.\\
The loop of mutation - crossover - evaluation is repeated until a termination condition is met.
\begin{figure}
    \includegraphics[width=\linewidth]{Images/mhdp_algo.png}
    \caption{MHDP algorithm}
    \label{fig:mhdp}
\end{figure}

\section{Implementation of MHDP}
The code for the implementation has been written in Python. As a requirement, some very popular Python modules have been used alongside the "genetic oriented" module \textit{inspyred}.
Concerning the implementation of the MHDP algorithm, we followed the pseudocodes present in the paper for : the generation of the initial population, the crossover operator
and the adjustment of infeasible solutions.
The mutation operation was easy to implement, we only needed to select two different random individuals from the pool of possible solutions and apply (\ref{eqn:pso}).
Reguarding the evaluation procedure, we had to implement also the 
\textit{fast non-dominated sort} algorithm to find the Pareto fronts.

\section{Results and analysis}
The algorithm has been tested upon the same data used by the researchers in \cite{fire_distinguish}. More precisely, the used data correspond to the fire occurred in Huzhong, region located in
Mt. Daxing’anling, at $10:40$ local time on June 29, 2010. The flames hit a total of 7 fire points in that specific area. \\
For the purposes of this work, the researchers reported all the necessary parameters to calculate the objective functions. The only element that's missing is the upper bound to the number
of vehicles for each fire point ($U_i$). Since in the paper it was presented as a given parameters, we used a representative number $10$ for all the points.\\
In Table \ref{tab:our_solutions} we reported the results obtained after running the algorithm 4 consecutive times. The lowest calculated time to extinguish all
the fires is $6.17$ hours, using all the vehicles, while the lowest number of vehicles used is $29$, that is the lower bound given by the authors. In the latter case, the extinguish
time is $40.04$ hours. In figure \ref{fig:pareto_results} these solutions are plot on a 2D graph with $f_1$ on the $x$ axis and $f_2$ on the $y$ axis. 
We can see that most of the solutions are concentrated between 5 and 20 hours for $f_1$ and these solutions cover almost all the values in $f_2$, so the results are satisfying. 
Our results are very similar to the ones found by the authors, however the pareto front in their experiment (Fig. \ref{fig:authors_results}) is very well distributed and 
all the runs produce very similar solutions.

\begin{figure}
    \includegraphics[width=\linewidth]{Images/our_results_4_runs.png}
    \caption{Our Pareto solutions}
    \label{fig:pareto_results}
\end{figure}

\begin{figure}
    \includegraphics[width=\linewidth]{Images/authors_results.png}
    \caption{Authors' Pareto solutions}
    \label{fig:authors_results}
\end{figure}

\section{Conclusion}
During the development of the project we had the opportunity to apply EC on a real world scenario 
and to implement EAs by hand to deeper understand their functioning. Moreover, we devised something that may be useful for others and may make the work
of firefighters more efficient.\\
While reading the chosen paper, we found some unclear explanations; however, after a quick reading of the papers referenced by the authors, we were able to understand better how to proceed.
Eventually, this work turned out to be very satisfactory. Indeed, we were able to reach significant results starting with minimal experience in genetic programming (as we all know, theorethical background sometimes doesn't help that much in practice) and we had the opportunity to experiment and verify the power of such bio-inspired algorithms.

\section{Appendix}
%\begin{center}
%    \begin{threeparttable}
%        \caption{The Skewing Angles ($\beta$) for $\fam0 Mu(H)+X_2$ and
%        $\fam0 Mu(H)+HX$~\tnote{a}}
%        \begin{tabular}{rlcc}
%            \hline
%            &   & $\fam0 H(Mu)+F_2$ & $\fam0 H(Mu)+Cl_2$ \\
%            \hline
%            &$\beta$(H)  & $80.9^\circ\tnote{b}$ & $83.2^\circ$ \\
%            &$\beta$(Mu) & $86.7^\circ$ & $87.7^\circ$ \\
%            \hline
%        \end{tabular}
%        \begin{tablenotes}
%            \item[a] for the abstraction reaction, $\fam0 Mu+HX \rightarrow MuH+X$.
%            \item[b] 1 degree${} = \pi/180$ radians.
%        \end{tablenotes}
%    \end{threeparttable}

%\end{center}


\begin{center}
    \begin{threeparttable}
    \caption{$k_S$ values of different fuel types}
        \begin{tabular}{cccc}
            \hline
            \thead{Forest\\ types} & \thead{Meadow (I)} & \thead{Secondary \\forest (II)} & \thead{Coniferous \\forest (III)}\\
            \hline
            $k_s$ & $1.0$ & $0.7$ & $0.4$\\
            \hline
        \end{tabular}
        \label{tab:ks}
    \end{threeparttable}

\end{center}



\begin{center}
    \begin{threeparttable}
        \caption{$v_w$ ($k_w = e^{0.1783v_w}$) value of different wind force}
        \begin{tabular}{cc}
            \hline
            \thead{Wind force\\ leve}l & $v_w(m/s)$\\
            \hline
            1 & 2 \\
            2 & 3.6\\
            3 & 5.4\\
            4 & 7.4\\
            5 & 9.8\\
            6 & 12.3\\
            7 & 14.9\\
            8 & 17.7\\
            9 & 20.8\\
            10 & 24.2\\
            11 & 27.8\\
            12 & 29.8\\
            \hline
        \end{tabular}
        \label{tab:vw}
    \end{threeparttable}

\end{center}


\begin{center}
    \begin{threeparttable}
        \caption{$k_{\varphi}$ value of different terrain slopes}
        \begin{tabular}{cc}
            \hline
            Slope range & $k_{\varphi}$\\
            \hline
            $-42\degree \sim -38\degree$ & $0.007$\\
            $-37\degree \sim -33\degree$ & $0.13$\\
            $-32\degree \sim -28\degree$ & $0.21$\\
            $-27\degree \sim -23\degree$ & $0.32$\\
            $-22\degree \sim -18\degree$ & $0.46$\\
            $-17\degree \sim -13\degree$ & $0.63$\\
            $-12\degree \sim -8\degree$ & $0.83$\\
            $-7\degree \sim -3\degree$ & $0.90$\\
            $-2\degree \sim 2\degree$ & $1.00$\\
            $3\degree \sim 7\degree$ & $1.20$\\
            $8\degree \sim 12\degree$ & $1.60$\\
            $13\degree \sim 17\degree$ & $2.10$\\
            $18\degree \sim 22\degree$ & $2.90$\\
            $23\degree \sim 27\degree$ & $4.10$\\
            $28\degree \sim 32\degree$ & $6.20$\\
            $33\degree \sim 37\degree$ & $10.10$\\
            $38\degree \sim 42\degree$ & $17.50$\\
            \hline
        \end{tabular}
        \label{tab:kphi}
    \end{threeparttable}

\end{center}


\begin{table}
    \centering
    \caption{PRODUCED PARETO SOLUTIONS}
    \renewcommand{\arraystretch}{1.5}
    \begin{tabular}{ |c c c c c|  }
        \hline
        Runs & Solution Number & Scheduling schemes & $f_1$ (h) & $f_2$\\
        \hline
        \multirow{9}{*}{1} & 1 & $\{5, 4, 3, 8, 7, 6, 4\}$ & 8.25  & 37 \\
                           & 2 & $\{7, 3, 4, 8, 8, 6, 4\}$ & 6.17  & 40 \\
                           & 3 & $\{5, 3, 3, 7, 6, 5, 6\}$ & 11.83 & 35 \\
                           & 4 & $\{5, 3, 3, 7, 7, 6, 3\}$ & 12.76 & 34 \\
                           & 5 & $\{5, 2, 3, 8, 6, 4, 3\}$ & 33.38 & 31 \\
                           & 6 & $\{5, 3, 3, 7, 8, 6, 4\}$ & 9.06  & 36 \\
                           & 7 & $\{7, 3, 3, 8, 7, 6, 4\}$ & 7.31  & 38 \\
                           & 8 & $\{5, 2, 3, 6, 7, 5, 4\}$ & 19.33 & 32 \\
                           & 9 & $\{6, 2, 3, 7, 7, 5, 3\}$ & 15.87 & 33 \\
        \hline
        \multirow{12}{*}{2} & 1  & $\{6, 4, 3, 8, 7, 6, 4\}$ & 7.28  & 38 \\
                           & 2  & $\{6, 3, 3, 8, 8, 6, 5\}$ & 6.71  & 39 \\
                           & 3  & $\{6, 3, 3, 9, 7, 7, 5\}$ & 6.46  & 40 \\
                           & 4  & $\{5, 2, 3, 6, 6, 4, 3\}$ & 40.04 & 29 \\
                           & 5  & $\{5, 2, 3, 7, 6, 5, 3\}$ & 18.68 & 31 \\
                           & 6  & $\{5, 3, 4, 7, 6, 6, 4\}$ & 10.76 & 35 \\
                           & 7  & $\{5, 2, 3, 7, 6, 5, 4\}$ & 15.48 & 32 \\
                           & 8  & $\{5, 3, 3, 7, 8, 6, 5\}$ & 8.67  & 37 \\
                           & 9  & $\{5, 2, 3, 6, 6, 5, 3\}$ & 24.36 & 30 \\
                           & 10 & $\{5, 2, 3, 7, 7, 5, 5\}$ & 13.26 & 34 \\
                           & 11 & $\{5, 2, 3, 7, 7, 5, 4\}$ & 13.65 & 33 \\
                           & 12 & $\{5, 3, 3, 7, 7, 5, 6\}$ & 10.00 & 36 \\
        \hline
        \multirow{10}{*}{3} & 1  & $\{6, 4, 3, 8, 7, 6, 5\}$ & 6.89  & 39 \\
                           & 2  & $\{5, 3, 3, 7, 7, 5, 3\}$ & 13.74 & 33 \\
                           & 3  & $\{5, 2, 3, 6, 6, 5, 3\}$ & 24.36 & 30 \\
                           & 4  & $\{5, 3, 4, 7, 7, 6, 4\}$ & 8.93 & 36 \\
                           & 5  & $\{5, 2, 3, 8, 7, 5, 4\}$ & 12.66 & 34 \\
                           & 6  & $\{5, 3, 3, 9, 7, 7, 4\}$ & 7.82 & 38 \\
                           & 7  & $\{5, 2, 4, 6, 6, 5, 3\}$ & 23.73 & 31 \\
                           & 8  & $\{5, 2, 3, 8, 8, 5, 4\}$ & 12.16 & 35 \\
                           & 9  & $\{6, 3, 5, 8, 7, 6, 5\}$ & 6.38 & 40 \\
                           & 10 & $\{5, 4, 4, 7, 7, 6, 4\}$ & 8.60 & 37 \\
        \hline
        \multirow{10}{*}{4} & 1  & $\{5, 2, 3, 7, 6, 5, 3\}$ & 18.68  & 31 \\
                           & 2  & $\{5, 3, 3, 7, 6, 6, 3\}$ & 14.60 & 33 \\
                           & 3  & $\{5, 3, 3, 8, 7, 5, 4\}$ & 9.56 & 35 \\
                           & 4  & $\{6, 3, 4, 8, 8, 5, 4\}$ & 7.45 & 38 \\
                           & 5  & $\{5, 4, 3, 7, 7, 7, 4\}$ & 8.88 & 37 \\
                           & 6  & $\{6, 3, 3, 7, 8, 5, 4\}$ & 9.06 & 36 \\
                           & 7  & $\{6, 4, 3, 7, 8, 6, 5\}$ & 7.37 & 39 \\
                           & 8  & $\{6, 3, 3, 9, 8, 6, 5\}$ & 6.31 & 40 \\
                           & 9  & $\{5, 2, 3, 7, 7, 5, 5\}$ & 13.26 & 34 \\
                           & 10 & $\{5, 2, 3, 7, 6, 5, 4\}$ & 15.48 & 32 \\
        \hline
    \end{tabular}
    
    \label{tab:our_solutions}
\end{table}

\begin{thebibliography}{1}

\bibitem{IEEEhowto:kopka}
H.~Kopka and P.~W. Daly, \emph{A Guide to \LaTeX}, 3rd~ed.\hskip 1em plus
  0.5em minus 0.4em\relax Harlow, England: Addison-Wesley, 1999.

\bibitem{fire_distinguish}Tian, G., Ren, Y. \& Zhou, M. Dual-Objective Scheduling of Rescue Vehicles to Distinguish Forest Fires via Differential Evolution and Particle Swarm Optimization Combined Algorithm. {\em IEEE Transactions On Intelligent Transportation Systems}. \textbf{17}, 3009-3021 (2016)

\end{thebibliography}


\end{document}


