\documentclass[journal]{IEEEtran}
% *** CITATION PACKAGES ***
\usepackage{cite}

% *** GRAPHICS RELATED PACKAGES ***
%
\usepackage[pdftex]{graphicx}

% *** MATH PACKAGES ***
\usepackage{amsmath}
\interdisplaylinepenalty=2500

% *** SPECIALIZED LIST PACKAGES ***
% to write algorithms in the corpus of the report
\usepackage{algorithmic}


% *** PDF, URL AND HYPERLINK PACKAGES ***
\usepackage{url}

% correct bad hyndenation here
\hyphenation{op-tical net-works semi-conduc-tor}


\begin{document}
\title{Dual-Objective Scheduling of Rescue Vehicles to\\
Distinguish Forest Fires via Differential Evolution\\
and Particle Swarm Optimization\\
Combined Algorithm}

\author{Guidolin~Davide,
        Guglielmi~Matteo}% 

% The paper headers
\markboth{University of Trento, Bio-Inspired AI, September~6}%
{Shell \MakeLowercase{\textit{et al.}}: Implementation of the paper "Dual-Objective Scheduling of Rescue Vehicles to 
Distinguish Forest Fires via Differential Evolution
and Particle Swarm Optimization
Combined Algorithm"}
% make the title area
\maketitle

% As a general rule, do not put math, special symbols or citations
% in the abstract or keywords.
\begin{abstract}
With the increasing issue of global warming, the problem of forest fires during summer seasons is becoming more severe every year.
For this reason we decided to focus our attention on a project that could possibly deal with this problem. Our attention landed on the paper 
\textit{"Dual-Objective Scheduling of Rescue Vehicles to Distinguish Forest Fires via Differential Evolution and Particle Swarm Optimization Combined Algorithm"}
written by \textit{Guangdong Tian, Yaping Ren, and MengChu Zhou, Fellow, IEEE}. 
In this paper the authors present a method to optimize the fire distinguish time and the number of vehicles used to distinguish a set of fires.
Their approach is applied to a real-world scenario in Mt. Daxing’anling, China.
\end{abstract}

% Note that keywords are not normally used for peerreview papers.
\begin{IEEEkeywords}
    PSO, DE, NSGA-II, Pareto Solutions, Genetic Operators, MHDP
\end{IEEEkeywords}

\IEEEpeerreviewmaketitle

\section{Introduction}
The problem of forest fires is becoming a big issue all around the world. With the continuous rise in temperature and with the less frequent rains in summer, the number of forest fires is increasing every year. However, the number of rescue vehicles is limited and, in case of multiple fire points, deciding how many vehicles to use for each fire point is a difficult task that has to be solved very quickly. In particular, different fire points may have different weather characteristics, like the temperature and the wind speed, and different terrain characteristics, like the slope and the type of terrain, and these parameters have to be taken into account during the decision of the number of vehicles for each fire point. Finally, the distance of the fire point to the fire department and the time that each vehicle takes to extinguish a fire are very important parameters.\\
In the paper \textit{Dual-Objective Scheduling of Rescue Vehicles to Distinguish Forest Fires via Differential Evolution and Particle Swarm Optimization Combined Algorithm}\cite{fire_distinguish}, by Tian, G., Ren, Y. \& Zhou, M., the authors present a  Multi-objective Hybrid Differential-evolution and Particle-swarm-optimization (MHDP) algorithm to minimize the time spent to extinguish a fire while minimizing the total number of vehicles used.

\begin{thebibliography}{1}

\bibitem{IEEEhowto:kopka}
H.~Kopka and P.~W. Daly, \emph{A Guide to \LaTeX}, 3rd~ed.\hskip 1em plus
  0.5em minus 0.4em\relax Harlow, England: Addison-Wesley, 1999.

\end{thebibliography}


\end{document}


