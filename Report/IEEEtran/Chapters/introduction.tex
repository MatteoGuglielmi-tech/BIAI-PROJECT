\section{Introduction}
The problem of forest fires is becoming a big issue all around the world. With the continuous rise in temperature and with the less frequent rains in summer, the number of forest fires is increasing every year. However, the number of rescue vehicles is limited and, in case of multiple fire points, deciding how many vehicles to use for each fire point is a difficult scheduling task that has to be solved very quickly. 
In particular, different fire points may have different weather characteristics (e.g. the temperature and the wind speed) and/or varying terrain characteristics such as different slopes and terrain types; these parameters have to be taken into account during the decision making to achieve a proper solution.
Finally, other aspects to consider are : the distance of a fire point from the fire department and the time that each vehicle takes to extinguish a fire.\\
In the paper \textit{Dual-Objective Scheduling of Rescue Vehicles to Distinguish Forest Fires via Differential Evolution and Particle Swarm Optimization Combined Algorithm}\cite{fire_distinguish}, by Tian, G., Ren, Y. \& Zhou, M., the authors present a  Multi-objective Hybrid Differential-evolution and Particle-swarm-optimization (MHDP) algorithm to minimize the time spent to extinguish all the fires while minimizing the total number of vehicles used. The proposed algorithm integrates differential evolution (DE) and particle swarm optimization (PSO) into a multi-objective optimization algorithm in order to increase the population diversity with DE and improving the convergence ability with PSO.\\
This paper is organized as follows: Section II describes the problem and the objectives that has to be minimized. Section III describes the MHDP algorithm. Section IV discuss implementation aspects. Section V shows and analyzes the results.
Finally, Section VI concludes our work with ending considerations.
